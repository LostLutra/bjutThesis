\chapter{绪\quad 论} \chaptermark{绪\quad 论}\label{chap:introduction}

\section{研究意义}

机器人发明于上个世纪50年代,其目标是满足人类日常生活、工作、生产的各种需求。通过近几十年的研究,以机械臂为代表的工业机器人相关技术日趋成熟,并已广泛应用于人类工业化生产中,其可在生产线等固定工作空间代替人类完成重复、繁琐、危险或高精度的工业生产任务。随着科技的发展和人类生活水平的日益提高,人们对机器人的能力提出了越来越高的要求,赋予机器人的移动能力和智能,使其在不同工作空间代替人类完成更多更复杂的任务这一愿望越来越强烈,因此,移动机器人应运而生,其应用场景包括日常家庭服务、公共场所服务、医疗陪护、工业生产、农业耕作、救灾救援、勘探勘测等,其工作空间从人类日常活动范围扩展到人类无法到达的陆地、深海、太空和外星球等。

部分线性变系数模型作为一类广受欢迎的半参数模型,其理论性质已经得到 了深入的研究,研究成果也已较为成熟。基于空间数据和部分线性变系数模型所 具有的特点,本文拟将二者相结合,考虑部分线性变系数空间自回归模型,该模 型灵活性较强,包含了多种参数、非参数回归模型及空间自回归模型,适应性广 且可以更好地解决社会、经济等问题,故研究该模型的统计推断具有极其重要的 现实意义,考虑到该模型中空间滞后项的存在,以及空间相关的多方向性,传统 的部分线性变系数模型的估计方法并不能直接推广到部分线性变系数空间自回归 模型,因此,本文从现有理论出发,综合考虑部分线性变系数空间自回归模型,旨 在找到有效的估计方法,并证明参数估计的渐近性质,给出非参数分量最优的收 敛率。

\section{国内外研究现状}
\lipsum[1-4]
\section{研究内容}

\lipsum[8-12]
