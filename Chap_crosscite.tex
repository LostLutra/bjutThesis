\chapter{交叉引用} \chaptermark{绪\quad 论}\label{chap:crosscite}

\section{研究背景、意义及来源}

机器人发明于上个世纪50年代,其目标是满足人类日常生活、工作、生产的各种需求。通过近几十年的研究,以机械臂为代表的工业机器人相关技术日趋成熟,并已广泛应用于人类工业化生产中,其可在生产线等固定工作空间代替人类完成重复、繁琐、危险或高精度的工业生产任务。随着科技的发展和人类生活水平的日益提高,人们对机器人的能力提出了越来越高的要求,赋予机器人的移动能力和智能,使其在不同工作空间代替人类完成更多更复杂的任务这一愿望越来越强烈,因此,移动机器人应运而生,其应用场景包括日常家庭服务、公共场所服务、医疗陪护、工业生产、农业耕作、救灾救援、勘探勘测等,其工作空间从人类日常活动范围扩展到人类无法到达的陆地、深海、太空和外星球等。


\citet{hu_satellite-based_2019}

\citet{hu_estimating_2017}

\citet{descary_functional_2019}

\citet{hu_satellite-based_2019,hu_estimating_2017,feng_wild_2011}